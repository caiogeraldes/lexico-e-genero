% TEX program = xelatex
% TEX encoding = UTF-8

\documentclass[article, 12pt]{memoir}

\OnehalfSpacing%

\usepackage{silence}
\WarningsOff[biblatex]

\usepackage[no-math]{fontspec}
\usepackage[main=brazil]{babel}

\setmainfont{Gentium Plus}
\setmonofont[Scale=0.75]{Noto Mono}

\usepackage{indentfirst}
\usepackage{linguex}
\usepackage{tikz} %for all basic options
\usepackage{tikz-qtree} %for simple tree syntax
\usepgflibrary{arrows} %for arrow endings
\usetikzlibrary{positioning,shapes.multipart} %for structured nodes
\usetikzlibrary{tikzmark}
\usepackage{tikz-qtree}
\usepackage{tree-dvips}
\usepackage{xargs}
\usepackage{booktabs}
\usepackage{tabularx}
\usepackage[normalem]{ulem}

% Pacotes de diagramação
\usepackage{paralist}
\usepackage{longtable}
\usepackage{multirow}
\usepackage{amsmath}
\usepackage{graphicx}
\usepackage{hyphenat}

\usepackage{hyperref}
\usepackage{bookmark}

\author{Caio Borges Aguida Geraldes}
\title{Distribuição de verbos \emph{DcI} e \emph{GcI} em Grego Antigo: dialeto e gênero literário}
\date{\today}

\usepackage[backend=biber,
            style=abnt,
            pretty,
            repeatfields,
            noslsn,
            % natbib,
            extrayear,
            ]{biblatex}
\addbibresource{biblio.bib}

\usepackage{lipsum}
\begin{document}

\maketitle

\begin{abstract}
  \lipsum[1]
\end{abstract}

\chapter{Introdução}

Em trabalho anterior~\cite{Geraldes2020,Geraldes2021}, mostrei que a concordância de caso entre objeto indireto da matriz e predicado secundário de uma oração infinitiva (exemplificadas em~\ref{gloss:attrac}), estavam correlacionadas pelos seguintes fatores:
\begin{inparaenum}[(a)]
  \item distância entre controlador e alvo, quanto menor mais frequente;
  \item classe de verbo infinitivo, sobretudo ocorrendo com cópulas;
  \item classe do verbo matriz, sobretudo ocorrendo com verbos com sentido modal\slash{}deôntico;
  \item autor, sendo mais frequente em Platão do que em Xenofonte e praticamente inexistente em Heródoto.
\end{inparaenum}


\ex.\label{gloss:attrac}\ag.\label{elthonta}symbọːléw-ẹː \uwave{tɔ̂ːj Ksenopʰɔ̂ːnti} \uline{eltʰóntɑ} {ẹːs delpʰọːs} ɑnɑkojnoɔ̂ːsɑj {tɔ̂ːj tʰeɔ̂ːj} {peri tɛ̂ːs porẹ́ːɑs}\\
aconselha-\textsc{3sg} X.\textsc{dat.sg.m} indo-\textsc{acc.sg.m} para-Delfos interrogar.\textsc{inf} o-deus.\textsc{dat.sg} sobre-a-viagem\\
Ele aconselha Xenofonte ir a Delfos interrogar o deus sobre a viagem. (Xen. Anab. 3.1.5) \textbf{[Sem atração]}
\bg.\label{elthonti}ɑpʰɛ̂ːk-e \uwave{moj} \uline{eltʰónt-i} {pros hymɑ̂s} légẹːn tɑlɛːtʰɛ̂ː\\
permitiu-\textsc{3sg} \textsc{pron{(1sg.dat.sg)}} indo-\textsc{dat.sg.m} frente-a-vós dizer.\textsc{inf} a-verdade-\textsc{acc.pl.n}\\
Ele me permitiu ir frente a vós [e\slash{}para] dizer a verdade.  (Xen. Hell. 6.1.13) \textbf{[Com atração]}\footnote{Os exemplos utilizados foram retirados das edições disponibilizadas no TLG~\cite{tlg}. A transliteração foi realizada automaticamente utilizando o pacote \texttt{cltk}~\cite{johnson2014}, seguindo a reconstrução fonológica apresentada em~\textcite{Probert2010}.}


Defendi que, pelo menos do ponto de vista de uma teoria linguística, seria possível explicar que a correlação entre atração de caso e autoria não indicava necessariamente que a atração estaria condicionada por diferenças dialetais, posto que era esperado que a presença de verbos principais com valor modal de ``ser possível ou necessário a alguém fazer algo'' era presumivelmente maior em diálogos filosóficos (Platão) do que em narrativas históricas (Heródoto).
Essa pressuposição não é, no entanto, corroborada por nenhum estudo e a sua negativa é naturalmente possível: ``a distribuição de verbos modalizantes e não-modalizantes em grego antigo não depende do gênero literário.''

Se a hipótese positiva estiver correta, um classificador possível para gênero literário será a frequência relativa de verbos modalizantes entre os verbos com construção obliquo com infinitivo.

\chapter{Metodologia}
Plataforma: \textcite{stanza}

Modelo: \textcite{AGLDT2011,AGLDT2014}

\chapter{Análises}

\begin{itemize}
  \item oi.
  \item
\end{itemize}

\chapter{Conclusões}

\printbibliography%

\end{document}
